% Options for packages loaded elsewhere
\PassOptionsToPackage{unicode}{hyperref}
\PassOptionsToPackage{hyphens}{url}
%
\documentclass[
]{article}
\usepackage{amsmath,amssymb}
\usepackage{iftex}
\ifPDFTeX
  \usepackage[T1]{fontenc}
  \usepackage[utf8]{inputenc}
  \usepackage{textcomp} % provide euro and other symbols
\else % if luatex or xetex
  \usepackage{unicode-math} % this also loads fontspec
  \defaultfontfeatures{Scale=MatchLowercase}
  \defaultfontfeatures[\rmfamily]{Ligatures=TeX,Scale=1}
\fi
\usepackage{lmodern}
\ifPDFTeX\else
  % xetex/luatex font selection
\fi
% Use upquote if available, for straight quotes in verbatim environments
\IfFileExists{upquote.sty}{\usepackage{upquote}}{}
\IfFileExists{microtype.sty}{% use microtype if available
  \usepackage[]{microtype}
  \UseMicrotypeSet[protrusion]{basicmath} % disable protrusion for tt fonts
}{}
\makeatletter
\@ifundefined{KOMAClassName}{% if non-KOMA class
  \IfFileExists{parskip.sty}{%
    \usepackage{parskip}
  }{% else
    \setlength{\parindent}{0pt}
    \setlength{\parskip}{6pt plus 2pt minus 1pt}}
}{% if KOMA class
  \KOMAoptions{parskip=half}}
\makeatother
\usepackage{xcolor}
\usepackage[margin=1in]{geometry}
\usepackage{graphicx}
\makeatletter
\def\maxwidth{\ifdim\Gin@nat@width>\linewidth\linewidth\else\Gin@nat@width\fi}
\def\maxheight{\ifdim\Gin@nat@height>\textheight\textheight\else\Gin@nat@height\fi}
\makeatother
% Scale images if necessary, so that they will not overflow the page
% margins by default, and it is still possible to overwrite the defaults
% using explicit options in \includegraphics[width, height, ...]{}
\setkeys{Gin}{width=\maxwidth,height=\maxheight,keepaspectratio}
% Set default figure placement to htbp
\makeatletter
\def\fps@figure{htbp}
\makeatother
\setlength{\emergencystretch}{3em} % prevent overfull lines
\providecommand{\tightlist}{%
  \setlength{\itemsep}{0pt}\setlength{\parskip}{0pt}}
\setcounter{secnumdepth}{-\maxdimen} % remove section numbering
\ifLuaTeX
  \usepackage{selnolig}  % disable illegal ligatures
\fi
\IfFileExists{bookmark.sty}{\usepackage{bookmark}}{\usepackage{hyperref}}
\IfFileExists{xurl.sty}{\usepackage{xurl}}{} % add URL line breaks if available
\urlstyle{same}
\hypersetup{
  pdftitle={Exploratory Analysis},
  pdfauthor={James Santarpio and Mardi Elings},
  hidelinks,
  pdfcreator={LaTeX via pandoc}}

\title{Exploratory Analysis}
\author{James Santarpio and Mardi Elings}
\date{2023-11-30}

\begin{document}
\maketitle

\hypertarget{data-analytics-term-project}{%
\section{Data Analytics Term
Project}\label{data-analytics-term-project}}

\hypertarget{project-desription}{%
\subsection{Project Desription}\label{project-desription}}

\hypertarget{topic}{%
\subsubsection{Topic:}\label{topic}}

The effect of drug decriminalization in Oregon on overdose deaths and
well-being compared to California and Washington.

\hypertarget{why-it-matters}{%
\subsubsection{Why it Matters}\label{why-it-matters}}

The study of drug decriminalization and overdose rates in Oregon has
potential for significant policy and societal relevance. There are many
debates on the benefits and drawbacks that drug use and
decriminalization has on potential for policy implications involving
economics, public health, criminal justice, and community well-being.

\hypertarget{research-question-and-motivation}{%
\subsubsection{Research Question and
Motivation}\label{research-question-and-motivation}}

What is the effect of drug decriminalization on overdose deaths and
well-being? A policy change like Oregon's can have significant
consequences on populations who are most at risk of overdose deaths,
economic impacts on health care costs, and overall well-being.
Investigating the effects with county level data, we used a
difference-in-difference model with fixed effects to assess the impact
of the policy change in Oregon. By comparing the outcomes in Oregon
before and after the intervention to a control group of Washington and
California counties as control groups.

\hypertarget{description-of-data}{%
\subsubsection{Description of Data}\label{description-of-data}}

\hypertarget{overdose-mortality-by-county-in-the-us}{%
\paragraph{Overdose Mortality by County in the
US:}\label{overdose-mortality-by-county-in-the-us}}

The Overdose Mortality by County in the US is from the CDC Center for
Health Statistics for the years 2020-2023. These provisional
county-Level drug Overdose death counts are based on death records and
is received by the NCHS receives the death records from the state vital
registration offices through the Vital Statistics Cooperative Program
(VSCP). The provisional counts are reported by the county were the
decedent resided, not necessarily where they died.

\hypertarget{county-well-being-rankings}{%
\paragraph{County Well-Being
Rankings:}\label{county-well-being-rankings}}

The County Well-Being Rankings are for the years 2014-2023 and is
collected by County Health Rankings \& Roadmaps CHR\&R, a program of the
University of Wisconsin Population Health Institute. The health rankings
include deaths which include overdose deaths, years of potential life
lost, mental health statistics, income, and other health public health
rankings.

\hypertarget{overdose-by-drug}{%
\paragraph{Overdose by Drug:}\label{overdose-by-drug}}

The Overdose by drug dataset contains information on non-fatal overdose
data by state for 2018-2023. Provided by the CDC, DOSE (Drug Overdose
Surveillance and Epidemiology) collects health department reports in
each state.

Other variables of interest include opioid and stimulant overdoses and
the significance of the change from previous months in those overdoses.
Violent Crime Data From the FBI crime data reporter database, the number
of violent crimes in Oregon.

\hypertarget{how-are-they-related}{%
\paragraph{How are they related?}\label{how-are-they-related}}

The Well-Being Ranks, Overdose Mortality by County contain county fips
codes, county names and year in relation. The Overdose by Drug has a
relationship of total non-fatal drug overdoses in each state.

\hypertarget{data-processing}{%
\subsection{Data Processing}\label{data-processing}}

For the county health rankings data, we originally had years 2014-2023,
but after further inspection we found the way overdose deaths were
collected for 2014-2015 was significantly different than for the years
2016-2023. We opted to drop those years and due to those data sets not
including as many well-being measures.

The data collected was generally in clean condition. Some column name
changes were done, and general data manipulation was completed. For
example, health rankings came from yearly filet, which had to be merged
together. There was generally little missing data. Violent crime and
overdose by drugs contained zero missing values. However, a key data set
used, health rankings, contained 13 percent N/A values. These appeared
generally randomly scatter, and we will assume these are missing at
random and will be ignored.

\hypertarget{findings-and-visualizations}{%
\subsubsection{Findings and
Visualizations}\label{findings-and-visualizations}}

Initial findings include general decrease in well being, after 2020
which is likely due to Covid-19. Well-being variables include, mental
health days, fair or poor health days, excessive drinking, and income.
Additionally, drug overdose deaths have been increasing, especially
since 2020. However, it appears that Oregon's increase may be slower
relative to Washington and Oregon after decriminalizing drugs.

We compared the parallel trends of overdose deaths per-capita in all
states. Pre-treatment shows (2016-2020) shows

\includegraphics{Draft1Rmd_files/figure-latex/unnamed-chunk-1-1.pdf}

\includegraphics{Draft1Rmd_files/figure-latex/unnamed-chunk-2-1.pdf}
\includegraphics{Draft1Rmd_files/figure-latex/unnamed-chunk-2-2.pdf}

Comparing the rate of drug mortality by county depicts increases in many
counties in every state we are examining.

\includegraphics{Draft1Rmd_files/figure-latex/Cleveland dot plots-1.pdf}
\includegraphics{Draft1Rmd_files/figure-latex/Cleveland dot plots-2.pdf}

\includegraphics{Draft1Rmd_files/figure-latex/CA Cleveland dot plots-1.pdf}

\hypertarget{methods}{%
\subsubsection{Methods}\label{methods}}

\hypertarget{results}{%
\subsubsection{Results}\label{results}}

Decriminalizing drugs in Oregon has slowed the increase in drug overdose
deaths.

\hypertarget{limitations}{%
\subsubsection{Limitations}\label{limitations}}

It is a challenge to find a single cause for an impact on drug
decriminalization and its relationship to drug overdose deaths. Since
multiple factors can contribute to drug-related outcomes. There has been
some evidence linked to the Covid-19 pandemic contributing to drug use,
well-being, and mental health. There is limited generalizability to this
study as findings from synthetic control analysis are specific to the
context of Oregon drug policies and may not be generalized to other
regions or policies. There is also limited data post treatment period (2
years). Additionally, the synthetic Oregon may not be a good fit. This
will be further analyzed throughout the semester. As well as continuing
to fully understand the statistical basis for synthetic control better.

\end{document}
